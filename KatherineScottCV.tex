%%%%%%%%%%%%%%%%%%%%%%%%%%%%%%%%%%%%%%%%%%%%%%%%%%%%%%%%%%%%%%%%%%%%%%%%
%%%%%%%%%%%%%%%%%%%%%% Simple LaTeX CV Template %%%%%%%%%%%%%%%%%%%%%%%%
%%%%%%%%%%%%%%%%%%%%%%%%%%%%%%%%%%%%%%%%%%%%%%%%%%%%%%%%%%%%%%%%%%%%%%%%

%%%%%%%%%%%%%%%%%%%%%%%%%%%%%%%%%%%%%%%%%%%%%%%%%%%%%%%%%%%%%%%%%%%%%%%%
%% NOTE: If you find that it says                                     %%
%%                                                                    %%
%%                           1 of ??                                  %%
%%                                                                    %%
%% at the bottom of your first page, this means that the AUX file     %%
%% was not available when you ran LaTeX on this source. Simply RERUN  %% 
%% LaTeX to get the ``??'' replaced with the number of the last page  %% 
%% of the document. The AUX file will be generated on the first run   %%
%% of LaTeX and used on the second run to fill in all of the          %%
%% references.                                                        %%
%%%%%%%%%%%%%%%%%%%%%%%%%%%%%%%%%%%%%%%%%%%%%%%%%%%%%%%%%%%%%%%%%%%%%%%%

%%%%%%%%%%%%%%%%%%%%%%%%%%%% Document Setup %%%%%%%%%%%%%%%%%%%%%%%%%%%%

% Don't like 10pt? Try 11pt or 12pt
\documentclass[10pt]{article}

% This is a helpful package that puts math inside length specifications
\usepackage{calc}
\usepackage{textcomp}
%\usepackage{palatino}
\usepackage[resetlabels]{multibib}
\usepackage[ManyBibs]{currvita}

% Layout: Puts the section titles on left side of page
\reversemarginpar

%
%         PAPER SIZE, PAGE NUMBER, AND DOCUMENT LAYOUT NOTES:
%
% The next \usepackage line changes the layout for CV style section
% headings as marginal notes. It also sets up the paper size as either
% letter or A4. By default, letter was used. If A4 paper is desired,
% comment out the letterpaper lines and uncomment the a4paper lines.
%
% As you can see, the margin widths and section title widths can be
% easily adjusted.
%
% ALSO: Notice that the includefoot option can be commented OUT in order
% to put the PAGE NUMBER *IN* the bottom margin. This will make the
% effective text area larger.
%
% IF YOU WISH TO REMOVE THE ``of LASTPAGE'' next to each page number,
% see the note about the +LP and -LP lines below. Comment out the +LP
% and uncomment the -LP.
%
% IF YOU WISH TO REMOVE PAGE NUMBERS, be sure that the includefoot line
% is uncommented and ALSO uncomment the \pagestyle{empty} a few lines
% below.
%

%% Use these lines for letter-sized paper
\usepackage[paper=letterpaper,
            %includefoot, % Uncomment to put page number above margin
            marginparwidth=1.2in,     % Length of section titles
            marginparsep=.05in,       % Space between titles and text
            margin=1in,               % 1 inch margins
            includemp]{geometry}

%% Use these lines for A4-sized paper
%\usepackage[paper=a4paper,
%            %includefoot, % Uncomment to put page number above margin
%            marginparwidth=30.5mm,    % Length of section titles
%            marginparsep=1.5mm,       % Space between titles and text
%            margin=25mm,              % 25mm margins
%            includemp]{geometry}

%% More layout: Get rid of indenting throughout entire document
\setlength{\parindent}{0in}

%% This gives us fun enumeration environments. compactitem will be nice.
\usepackage{paralist}

%% Reference the last page in the page number
%
% NOTE: comment the +LP line and uncomment the -LP line to have page
%       numbers without the ``of ##'' last page reference)
%
% NOTE: uncomment the \pagestyle{empty} line to get rid of all page
%       numbers (make sure includefoot is commented out above)
%
\usepackage{fancyhdr,lastpage}
\pagestyle{fancy}
%\pagestyle{empty}      % Uncomment this to get rid of page numbers
\fancyhf{}\renewcommand{\headrulewidth}{0pt}
\fancyfootoffset{\marginparsep+\marginparwidth}
\newlength{\footpageshift}
\setlength{\footpageshift}
          {0.5\textwidth+0.5\marginparsep+0.5\marginparwidth-2in}
\lfoot{\hspace{\footpageshift}%
       \parbox{4in}{\, \hfill %
                    \arabic{page} of \protect\pageref*{LastPage} % +LP
%                    \arabic{page}                               % -LP
                    \hfill \,}}

% Finally, give us PDF bookmarks
\usepackage{color,hyperref}
\definecolor{darkblue}{rgb}{0.0,0.0,0.3}
\hypersetup{colorlinks,breaklinks,
            linkcolor=darkblue,urlcolor=darkblue,
            anchorcolor=darkblue,citecolor=darkblue}

%%%%%%%%%%%%%%%%%%%%%%%% End Document Setup %%%%%%%%%%%%%%%%%%%%%%%%%%%%


%%%%%%%%%%%%%%%%%%%%%%%%%%% Helper Commands %%%%%%%%%%%%%%%%%%%%%%%%%%%%

% The title (name) with a horizontal rule under it
%
% Usage: \makeheading{name}
%
% Place at top of document. It should be the first thing.
\newcommand{\makeheading}[1]%
        {\hspace*{-\marginparsep minus \marginparwidth}%
         \begin{minipage}[t]{\textwidth+\marginparwidth+\marginparsep}%
                {\large \bfseries #1}\\[-0.15\baselineskip]%
                 \rule{\columnwidth}{1pt}%
         \end{minipage}}

% The section headings
%
% Usage: \section{section name}
%
% Follow this section IMMEDIATELY with the first line of the section
% text. Do not put whitespace in between. That is, do this:
%
%       \section{My Information}
%       Here is my information.
%
% and NOT this:
%
%       \section{My Information}
%
%       Here is my information.
%
% Otherwise the top of the section header will not line up with the top
% of the section. Of course, using a single comment character (%) on
% empty lines allows for the function of the first example with the
% readability of the second example.
\renewcommand{\section}[2]%
        {\pagebreak[2]\vspace{1.3\baselineskip}%
         \phantomsection\addcontentsline{toc}{section}{#1}%
         \hspace{0in}%
         \marginpar{
         \raggedright \scshape #1}#2}

% An itemize-style list with lots of space between items
\newenvironment{outerlist}[1][\enskip\textbullet]%
        {\begin{itemize}[#1]}{\end{itemize}%
         \vspace{-.6\baselineskip}}

% An environment IDENTICAL to outerlist that has better pre-list spacing
% when used as the first thing in a \section 
\newenvironment{lonelist}[1][\enskip\textbullet]%
        {\vspace{-\baselineskip}\begin{list}{#1}{%
        \setlength{\partopsep}{0pt}%
        \setlength{\topsep}{0pt}}}
        {\end{list}\vspace{-.6\baselineskip}}

% An itemize-style list with little space between items
\newenvironment{innerlist}[1][\enskip\textbullet]%
        {\begin{compactitem}[#1]}{\end{compactitem}}

% To add some paragraph space between lines.
% This also tells LaTeX to preferably break a page on one of these gaps
% if there is a needed pagebreak nearby.
\newcommand{\blankline}{\quad\pagebreak[2]}

%%%%%%%%%%%%%%%%%%%%%%%% End Helper Commands %%%%%%%%%%%%%%%%%%%%%%%%%%%

%%%%%%%%%%%%%%%%%%%%%%%%% Begin CV Document %%%%%%%%%%%%%%%%%%%%%%%%%%%%

\begin{document}

 
\makeheading{\huge{Katherine Ann Scott} \\ \large{~Curriculum Vitae}  }

\section{Contact Information}
%
% NOTE: Mind where the & separators and \\ breaks are in the following
%       table.
%
% ALSO: \rcollength is the width of the right column of the table 
%       (adjust it to your liking; default is 1.85in).
%
\newlength{\rcollength}\setlength{\rcollength}{1.85in}%
%
\begin{tabular}[t]{@{}p{\textwidth-\rcollength}p{\rcollength}}
\href{mailto:katherineAScott@gmail.com}{Katherine Ann Scott}           & \textit{Voice: } (734) 358-0175 \\
\href{http://goo.gl/maps/QI0cn}{1701 Dexter Ave.}                      & \textit {Web: }\href{http://www.kscottz.com}{www.kscottz.com} \\
\href{http://goo.gl/maps/QI0cn}{Ann Arbor, MI, 48103 USA}              & \textit{Twitter: }\href{https://twitter.com/kscottz}{@kscottz} \\
\href{mailto:katherineAScott@gmail.com}{ katherine.a.scott@gmail.com}  & \textit{Github: }\href{https://github.com/kscottz}{kscottz}\\
                                                                       
\end{tabular}


%
%\section{Citizenship}
%USA

\section{Research Interests}
Computer vision, augmented reality, active vision systems, single and multiple instance object recognition, 3D reconstruction from video, and robotics.  \\

\section{Education}
\href{http://www.engineering.columbia.edu/}{\textbf{Columbia University}} \\
\href{http://www.engineering.columbia.edu/}{\textbf{The Fu Foundation School of Engineering and Applied Sciences}} \\
New York, New York USA
\begin{outerlist}

\item[] Master of Science in 
        \href{http://www.cs.columbia.edu/education/ms/visionAndGraphics}
             {Computer Science} 
        			December 2011
        \begin{innerlist}
        \item Areas of Study: Computer Vision, Graphics, Machine Learning.\newline \newline
        \end{innerlist}
\end{outerlist}


\href{http://www.engin.umich.edu/}{\textbf{The University of Michigan}} \\
\href{http://www.engin.umich.edu/}{\textbf{College of Engineering}} \\
Ann Arbor, Michigan USA
\begin{outerlist}

\item[] Bachelor of Science and Engineering in
        \href{http://www.eecs.umich.edu/}
             {Computer Engineering} 
        			April 2005
        \begin{innerlist}
        \item Areas of Study: Computer Graphics, Operating Systems, Networking.  
        \end{innerlist}

\item[] Bachelor of Science and Engineering in
        \href{http://www.eecs.umich.edu/}
             {Electrical Engineering} 
        			April 2005
        \begin{innerlist}
        \item Areas of Study: MEMS Design, VLSI Design.  
        \end{innerlist}
\item[] Academic Minor in Mathematics 

\end{outerlist}

\section{Professional Experience}
\href{http://www.sightmachine.com/}{\textbf{SightMachine}} \\
Ann Arbor, Michigan \\
\textit{Vice President of Research and Development}%
        \hfill \textbf{May 2011 to May 2013}  \\

\blankline
\href{http://www.sightmachine.com/}{SightMachine} is a start-up I helped co-found. SightMachine uses open source technology to create \href{http://www.theatlantic.com/technology/archive/2012/09/the-internet-and-things-how-manufacturing-could-get-better-with-a-dose-of-networked-data/262621/}{computer vision quality control systems} for \href{http://radar.oreilly.com/2013/02/sight-machine-new-vision-in-old-industry.html}{manufacturing and other industries}. 
\\
\begin{outerlist}
\item Developed a significant portion of the \href{http://www.simplecv.org/}{SimpleCV} python library. SimpleCV is an open source python framework for quickly developing computer vision and image processing applications. 
\item Designed custom computer vision applications for industrial and manufacturing customers covering such topics as metrology, image recognition, temporal signal processing, and color space analysis. This process included everything from initial system design and specification all the way through final testing, validation, and deployment.
\item Built and managed the SimpleCV open source community. This included validating user pull requests and bug reports on \href{https://github.com/sightmachine/SimpleCV}{github}, answering question on our \href{http://help.simplecv.org/questions/}{help forum}, managing our \href{http://sourceforge.net/blog/simplecv-gsoc/}{Google Summer of Code students}, co-authoring our \href{http://shop.oreilly.com/product/0636920024057.do}{O'Reilly publication} on SimpleCV, and producing SimpleCV \href{http://www.youtube.com/watch?v=UZSm7Q2bZoc&}{talks} and \href{http://www.youtube.com/watch?v=UZSm7Q2bZoc&}{tutorials}.
\item Assisted in the development and maintenance of \href{http://demo.simplecv.org/}{SimpleCV.js} a CoffeeScript companion library that closes follows the functionality of SimpleCV but runs in the web browser. 
\end{outerlist}
\blankline



\href{http://www.cybernet.com/}{\textbf{Cybernet Systems Corporation}}\\
Ann Arbor, Michigan USA \\
\textit{Research Engineer}%
        \hfill \textbf{February 2005 to December 2010}
\begin{outerlist}
\item Worked as a lead engineer, project manager, and primary investigator on various projects and engaged in recruiting, customer relations, business development activities.  
\blankline

\item Wrote the proposal for, won, managed, and contributed significant engineering work to four Phase I Small Business Innovative Research (SBIR) grants and two Phase II SBIR grants totaling nearly two million dollars in research funds. These projects are summarized below.
\blankline
%\begin{innerlist}
\item \textbf{Live Augmented Reality Play (LARP) for Training }%
        \hfill \textbf{Phase I \& II}\\
        \textit{U.S. Army Research Development and Engineering Command (RDECom)} \\
        Developed a live fire training system to supplant existing military shoot houses by replacing traditional targets with synthetic augmented reality characters. This system uses a pose determination system comprised of multiple inertial measurement systems and video-based augmented reality techniques. The vision system uses traditional ``barcode'' fiducials as truth points to help estimate the positions of natural image features. Natural features are mapped between the barcode truth points to provide complete training area map coverage. The inertial measurement system is used to support the visual system when low lighting or fast motion prevent the acquisition of robust camera data.  
\blankline
\item \textbf{Vigilance: Active CCTV System }%
        \hfill \textbf{Phase I \& II}\\
        %%%%%%%%%%%%%%%%%%%%%%%% FIX TENSES
        \textit{U.S. Research Development and Engineering Command (RDECom)} \\
        Vigilance is a real-time, network-based, database driven system for monitoring secure facilities and roadsides for suspicious and hostile activity. For fixed cameras Vigilance uses a hybrid background modeling system that incorporates multi-layer background codebooks models and frame differencing. Feature tracking is assisted by per feature extended Kalman filtering. Once tracking is accomplished features are characterized as either humans, vehicles, or objects using boosted Haar-like classifiers and descriptive statistics like aspect ratio, size, and motion. Once features are classified the system then looks for anomalous events like loitering, the dropping off and picking up of objects, and motion that deviates from a statistical model. Data about the image features and actions within each scene is recorded in an XML schema, which is then uploaded with the image to a PostgresSQL database. This project was discussed in the \href{http://media.
.com/node/12465303/comments}{October 23rd, 2008 edition of The Economist Magazine}. 
\blankline
\item \textbf{Augmented Reality for Combat Life Saver Training}%
        \hfill \textbf{Phase I}\\
        \textit{Office of the Secretary of Defense - Telemedicine and Advanced Technology Research Center (TATRC)} \\
  			Working with Dr. \href{http://www.med.umich.edu/meded/about/profiles/Andreatta.htm}{Prof. Pamela Andreatta} at the University of Michigan Clinical Simulation Center, I researched and designed a low-cost medical mannequin system for the training of the three main causes of battlefield death (tension pneumothorax, hemorrhage, and obstructed air way / cricothyrotomy). The proposed system would use augmented reality technology to perform tracking of the medical mannequin while displaying training information and medical imagery to the trainee. In addition to the core augmented reality system, we also proposed a tool and hand tracking methodology that could recognize and evaluate common life saving tasks using multi-view shape context to analyze the  user's hand configuration and common tools. The hand configuration, pose, motion, and tool selected, in conjunction with the simulation context, were to be used within an unsupervised, boosted, learning system to perform objective assessment of the combat life saver's capabilities.
\blankline
\item \textbf{Augmented Reality Maintenance Assistant}%
        \hfill \textbf{Phase I}\\
        \textit{U.S. Marine Corps Systems Command (MARCORSYSCOM)} \\
				Using Cybernet's proprietary touch screen tablet PC platform SWMA I created an augmented reality maintenance device that projects existing interactive electronic technical manuals onto maintenance area of the Light Armored Vehicle 25 (LAV25). The system performs tracking within the LAV25 using optical fiducials, and provides annotation, repair, and condition based maintenance data overlaid on the touchscreen tablet's screen. Using the system a maintainer could point the tablet�s rear mounted camera at a vehicle are, view registered annotation data on a live video feed, and then access relevant technical manual data. The user could then perform repairs using the tablet�s integrated repair and diagnostic tools.
\blankline                       
%\end{innerlist} 
\item I have also worked on a variety of other projects including a computer vision controlled actuated helmet for head mounted displays, SocialSim, a pilot project to test the effect of serious gaming on education outcomes for the University of Georgia, a Lua scripted GUI for our internal render engine, and a DIS to HLA gateway to serve as a web portal to JSAF simulations. 
%\end{innerlist}
\end{outerlist}

%%%%%%%%%%%%%%%%%%%%%%%%%%%%%%%%% Page break
\section{Academic Experience}
\href{http://graphics.cs.columbia.edu/top.html}{\textbf{Computer Graphics and User Interface Laboratory}} \\
Columbia University, New York, New York \\
\begin{outerlist}
\item[] \textit{Research Assistant}%
        \hfill \textbf{January 2011 to December 2011}
\begin{innerlist}
\item Under the direction of Prof. Steve Feiner assisted in the design and implementation of a collaborative augmented reality system for system maintenance and repair using the Goblin XNA architecture for the Raytheon corporation. The system allows subject matter experts to view the workspace of novice users and direct successful task completion using augmented reality visual queues. \\
\end{innerlist}
\end{outerlist}
\blankline
\\
\href{http://engin.umich.edu}{\textbf{The University of Michigan}}, 
Ann Arbor, Michigan USA
\begin{outerlist}
\item[] \textit{Undergraduate Research Assistant}%
        \hfill \textbf{August 2002 to April 2005} \\
        \href{http://nelab.engin.umich.edu/}{\textit{Neural Engineering Laboratory}}
\begin{innerlist}          
\item Conceived and designed a software modeling system for the simulation of neurochemical
diffusion in the brain. The system includes a scriptable differential equations solver, a random
walk diffusion model, a Matlab data fitting interface, and an OpenGL model visualization
utility. The end goal of this simulation system was the real-time spatial isolation of dopamine producing areas within live animals. \\
%\item I assisted in the development and characterization of novel MEMs based neurochemical
%sensors with integrated electrical recording sites including probe treatment, characterization, and design. \\

\end{innerlist}

\item[] \textit{Undergraduate Research Assistant}%
        \hfill \textbf{September 2000 to August 2002} \\
        \href{http://ai.eecs.umich.edu/RHex/ProjectOverview.html}{\textit{RHex Robotics Group}}
\begin{innerlist}         
\item Contributed in the development of an automated tuning platform for the RHex hexapod robot.
This software package tripled the energy efficiency of the robot, and greatly reduced the time
required in tuning the robot's gait parameters. This research culminated in an acknowledgment
in the 2004 IEEE International Conference on Robotics and Automation paper Automated
Gate Adaptation for Legged Robots\\
\item Created a client/server application for remote data logging and processing of robotic gate parameters within Matlab. This application used UDP and Matlab C-Mex subroutines. I also oversaw many of the groups interactions with the public as well as the student body.\\ 
\item Under the instruction of Professor Daniel Koditscheck and Professor Thomas Moore I completed research in the locomotion control of G. portentosa, including animal preparation and performance characterization\\. 
        
\end{innerlist}
\end{outerlist}

\blankline
%%%%%%%%%%% FIX LINE BREAK

\blankline
\href{http://www.vcu.edu}{\textbf{Virginia Commonwealth University}}, 
Richmond, Virginia USA
\begin{outerlist}
\item[] \textit{Undergraduate Research Assistant}%
        \hfill \textbf{May 2001 to August 2001} \\
        \href{http://www.vcu.edu/csbc/bbsi/people/faculty/anthony_guiseppi_elie.html}{\textit{NSF Research Experience for Undergraduates Program}}
\begin{innerlist}          
\item Overhauled a legacy DNA synthesis machine to proper working order by replacing the internal microfluidics system. \\
\item Participated in research into self-assembling monolayers for use in DNA detection technologies. My tasks
included monolayer depositions and characterization using quartz crystal microgravimetry. \\
\end{innerlist}
\end{outerlist}


\section{Recognition \& Presentations}
\begin{outerlist}
\item 2013 PyCon \href{http://www.youtube.com/watch?v=UZSm7Q2bZoc&}{talk} and \href{http://www.youtube.com/watch?v=UZSm7Q2bZoc&}{tutorial} presentation.
\item 2012 \href{http://www.nycresistor.com/2012/02/29/tickets-still-available-for-this-amazing-new-class-this-sunday/}{NYC Resistor SimpleCV Class}.
\item 2011 \href{http://vimeo.com/28723189#t=4534}{New York Tech Meetup} Presenter.
\item 2011 City College of New York \href{http://entrepreneurship.ccny.cuny.edu/kaylieprize}{Kaylie Prize} Recipient.
\item 2003 President RHex Robot Student Group
\item 2001 NSF-REU Student
\item 1997 Youth For Understanding Polish-American Exchange Student
\end{outerlist}



\section{Publications}
\begin{outerlist}
\item DeMaagd, Oliver, Oostendorp, and Katherine Scott \href{http://shop.oreilly.com/product/0636920024057.do}{\textit{Practical Computer Vision with SimpleCV}}, Cambridge: O'Reilly, 2012. Print. 

\item Scott, Katherine A., Dean, Frank. Haanpaa, Doug. Todd, James. ``Sensor Fusion for Live Training Augmented Reality.''  \textit{2008 Simulation Interoperability Standards Organization's Simulation Interoperability Workshop}. Orlando, FL. September 15-19, 2008.

\item Cohen, Charles J., Frank Morelli, Katherine Scott,``A Surveillance system for the Recognition of Intent within Individuals and Crowds.'' \textit{2008 IEEE Conference on Technologies for Homeland Security.} Waltham, MA. May 12-13, 2008.

\item Hay, Ron, Katherine Scott, Charles J. Cohen, ``Simulations as an Educational Environment for Balancing Disparate Needs.''  \textit{2006 Huntsville Simulation Conference.} Huntsville AL.  October 17, 2006 to October 19, 2006.

\item Hay, Ron, Katherine Scott, Charles J. Cohen.  ``Simulations as an Educational Environment for Balancing Disparate Needs.''  \textit{2006 Simulation Interoperability Standards Organization's Simulation Interoperability Workshop},  Orlando, FL, September 10-15, 2006.

\item Johnson M.D., Franklin R.K., Scott K.A., Brown R.B.,  Kipke D.R. ``Neural probes for concurrent detection of neurochemical and electrophysiological signals in vivo.''  \textit{ Proceedings of the 27th Annual International Conference of the IEEE Engineering in Medicine and Biology Society.}

\item Franklin, R.K.   Johnson, M.D.   Scottt, K.A.   Jun Ho Shim   Hakhyun Nam   Kipket, D.R.   Brown, R.B, ``Iridium oxide reference electrodes for neurochemical sensing with MEMS microelectrode arrays.'' \textit{The 4th IEEE Conference on Sensors.} Oct. 31 � Nov. 3, 2005, Irvine, CA, US.

\item Johnson M.D., Franklin R.K., Scott K.A., Brown R.B., Kipke D.R.  ``Neurochemical sensing with MEMS-based microelectrode arrays.''  Poster presented at WIMS fall 2004 conference, Ann Arbor, USA, October 22, 2004.

\item Johnson M.D., Scott K.A., Kipke D.R.  ``Hybrid neural implant systems: the chemical interface.''  Poster presented at WIMS spring 2004 conference, Ann Arbor, USA, May 2004.

\item Joel D. Weingarten, Gabriel A. D. Lopes, Martin Buehler, Richard E. Groff and Daniel E. Koditschek, ``Automated Gait Adaptation for Legged Robots'', \textit{IEEE International Conference on Robotics and Automation.} New Orleans, USA, April 2004.
\end{outerlist}

\section{Patents Submitted}
\begin{outerlist}
\item Foulk, Eugene. Hay, Ronald. Scott, Katherine.  Squiers, Merrill D.  Tesar, Joseph.  Cohen, Charles J.  Jacobus  Charles J.``Method for Controlling a GUI for Touchscreen Enabled Computers'' Patent 12/131,375 June-2-2008.
\item Scott, Katherine A. Haanpaa, Douglas P. Jacobus  Charles J. ``Augmented Reality for Equipment Maintainers.'' Patent 12/478.526. June-4-2009.
\item Scott, Katherine A. Haanpaa, Douglas P. Jacobus  Charles J. ``Automatic Fiducial Location and Orientation Estimation Using a Single
Truth Point.'' Patent 12/546,266. August-24-2009.
\end{outerlist}
\section{Technical Skills} 
\begin{lonelist}
\item[] \textbf{Languages:} \\
 Python, C++, C,  C\#, Java, Lua, Common Lisp, Coffee Script, UNIX shell scripting, MySQL.

\item[] \textbf{Libraries:} \\
OpenCV, OpenGL, Numpy, Scipy, SimpleCV, SciKits, OGRE, Boost, TinyXML, COM, MFC, GoblinXNA, Android SDK, and many more.

\item[] \textbf{Tools and Environments:} \\
         Unix, Linux, Windows, OS X, CVS, SVN, Trac, Git, Make, Visual Studio, NetBeans, Eclipse, EMacs, XEMacs, Matlab, and iPython Notebooks. 

\item[] \textbf{Applications:} \\
				\TeX{}, \LaTeX{}, B\textsc{ib}\TeX{}, Microsoft Office,
        and other common productivity packages for Windows, OS X, and
        Linux platforms
\end{lonelist}

\section{Interests \& Hobbies}
\begin{outerlist}
\item I am an avid gardener with an interest in \href{http://www.youtube.com/watch?v=KUyKp_cIFmc}{aquariums} and aquaculture, and plant collecting. I volunteered for five years with Ann Arbor's \href{http://www.projectgrowgardens.org/}{Project Grow Community Gardens}, and for two years with the \href{https://maps.google.com/maps/ms?ie=UTF8&oe=UTF8&msa=0&msid=203025543716789179192.000450c62f0577e61522e}{Serenity Community Garden} in Harlem, New York. 
\item For the \href{http://www.youtube.com/watch?v=FhBnCmBvLWo}{2012-2013 competition season}, I was a programming and electrical engineering mentor for \href{http://frcteam830.org/}{FIRST Robotics Competition Team 830}. 
\item I spent five years living in and organizing \href{http://arborwiki.org/Arbor_Vitae}{Arbor Vitae}, a fifty year old historic cooperative loft in Ann Arbor, Michigan. Arbor Vitae currently houses one of the largest libraries of world peace and peace activism materials, and frequently hosts community arts events.   
\end{outerlist}



%\center{ \textit{ References available upon request. }}
\end{document}


%%%%%%%%%%%%%%%%%%%%%%%%%% End CV Document %%%%%%%%%%%%%%%%%%%%%%%%%%%%%
