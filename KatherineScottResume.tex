%%%%%%%%%%%%%%%%%%%%%%%%%%%%%%%%%%%%%%%%%%%%%%%%%%%%%%%%%%%%%%%%%%%%%%%%
%%%%%%%%%%%%%%%%%%%%%% Simple LaTeX CV Template %%%%%%%%%%%%%%%%%%%%%%%%
%%%%%%%%%%%%%%%%%%%%%%%%%%%%%%%%%%%%%%%%%%%%%%%%%%%%%%%%%%%%%%%%%%%%%%%%

%%%%%%%%%%%%%%%%%%%%%%%%%%%%%%%%%%%%%%%%%%%%%%%%%%%%%%%%%%%%%%%%%%%%%%%%
%% NOTE: If you find that it says                                     %%
%%                                                                    %%
%%                           1 of ??                                  %%
%%                                                                    %%
%% at the bottom of your first page, this means that the AUX file     %%
%% was not available when you ran LaTeX on this source. Simply RERUN  %% 
%% LaTeX to get the ``??'' replaced with the number of the last page  %% 
%% of the document. The AUX file will be generated on the first run   %%
%% of LaTeX and used on the second run to fill in all of the          %%
%% references.                                                        %%
%%%%%%%%%%%%%%%%%%%%%%%%%%%%%%%%%%%%%%%%%%%%%%%%%%%%%%%%%%%%%%%%%%%%%%%%

%%%%%%%%%%%%%%%%%%%%%%%%%%%% Document Setup %%%%%%%%%%%%%%%%%%%%%%%%%%%%

% Don't like 10pt? Try 11pt or 12pt
\documentclass[9pt]{article}

% This is a helpful package that puts math inside length specifications
\usepackage{calc}
\usepackage{textcomp}
%\usepackage{palatino}
\usepackage[resetlabels]{multibib}
\usepackage[ManyBibs]{currvita}

% Layout: Puts the section titles on left side of page
\reversemarginpar

%
%         PAPER SIZE, PAGE NUMBER, AND DOCUMENT LAYOUT NOTES:
%
% The next \usepackage line changes the layout for CV style section
% headings as marginal notes. It also sets up the paper size as either
% letter or A4. By default, letter was used. If A4 paper is desired,
% comment out the letterpaper lines and uncomment the a4paper lines.
%
% As you can see, the margin widths and section title widths can be
% easily adjusted.
%
% ALSO: Notice that the includefoot option can be commented OUT in order
% to put the PAGE NUMBER *IN* the bottom margin. This will make the
% effective text area larger.
%
% IF YOU WISH TO REMOVE THE ``of LASTPAGE'' next to each page number,
% see the note about the +LP and -LP lines below. Comment out the +LP
% and uncomment the -LP.
%
% IF YOU WISH TO REMOVE PAGE NUMBERS, be sure that the includefoot line
% is uncommented and ALSO uncomment the \pagestyle{empty} a few lines
% below.
%

%% Use these lines for letter-sized paper
\usepackage[paper=letterpaper,
            %includefoot, % Uncomment to put page number above margin
            marginparwidth=1.00in,     % Length of section titles
            marginparsep=.05in,       % Space between titles and text
            margin=0.75in,               % 1 inch margins
            includemp]{geometry}

%% Use these lines for A4-sized paper
%\usepackage[paper=a4paper,
%            %includefoot, % Uncomment to put page number above margin
%            marginparwidth=30.5mm,    % Length of section titles
%            marginparsep=1.5mm,       % Space between titles and text
%            margin=25mm,              % 25mm margins
%            includemp]{geometry}

%% More layout: Get rid of indenting throughout entire document
\setlength{\parindent}{0in}

%% This gives us fun enumeration environments. compactitem will be nice.
\usepackage{paralist}

%% Reference the last page in the page number
%
% NOTE: comment the +LP line and uncomment the -LP line to have page
%       numbers without the ``of ##'' last page reference)
%
% NOTE: uncomment the \pagestyle{empty} line to get rid of all page
%       numbers (make sure includefoot is commented out above)
%
\usepackage{fancyhdr,lastpage}
%\pagestyle{fancy}
\pagestyle{empty}      % Uncomment this to get rid of page numbers
\fancyhf{}\renewcommand{\headrulewidth}{0pt}
\fancyfootoffset{\marginparsep+\marginparwidth}
\newlength{\footpageshift}
\setlength{\footpageshift}
          {0.5\textwidth+0.5\marginparsep+0.5\marginparwidth-2in}
\lfoot{\hspace{\footpageshift}%
       \parbox{4in}{\, \hfill %
                    \arabic{page} of \protect\pageref*{LastPage} % +LP
%                    \arabic{page}                               % -LP
                    \hfill \,}}

% Finally, give us PDF bookmarks
\usepackage{color,hyperref}
\definecolor{darkblue}{rgb}{0.0,0.0,0.3}
\hypersetup{colorlinks,breaklinks,
            linkcolor=darkblue,urlcolor=darkblue,
            anchorcolor=darkblue,citecolor=darkblue}

%%%%%%%%%%%%%%%%%%%%%%%% End Document Setup %%%%%%%%%%%%%%%%%%%%%%%%%%%%


%%%%%%%%%%%%%%%%%%%%%%%%%%% Helper Commands %%%%%%%%%%%%%%%%%%%%%%%%%%%%

% The title (name) with a horizontal rule under it
%
% Usage: \makeheading{name}
%
% Place at top of document. It should be the first thing.
\newcommand{\makeheading}[1]%
        {\hspace*{-\marginparsep minus \marginparwidth}%
         \begin{minipage}[t]{\textwidth+\marginparwidth+\marginparsep}%
                {\large \bfseries #1}\\[-0.15\baselineskip]%
                 \rule{\columnwidth}{1pt}%
         \end{minipage}}

% The section headings
%
% Usage: \section{section name}
%
% Follow this section IMMEDIATELY with the first line of the section
% text. Do not put whitespace in between. That is, do this:
%
%       \section{My Information}
%       Here is my information.
%
% and NOT this:
%
%       \section{My Information}
%
%       Here is my information.
%
% Otherwise the top of the section header will not line up with the top
% of the section. Of course, using a single comment character (%) on
% empty lines allows for the function of the first example with the
% readability of the second example.
\renewcommand{\section}[2]%
        {\pagebreak[2]\vspace{1.3\baselineskip}%
         \phantomsection\addcontentsline{toc}{section}{#1}%
         \hspace{0in}%
         \marginpar{
         \raggedright \scshape #1}#2}

% An itemize-style list with lots of space between items
\newenvironment{outerlist}[1][\enskip\textbullet]%
        {\begin{itemize}[#1]}{\end{itemize}%
         \vspace{-.6\baselineskip}}

% An environment IDENTICAL to outerlist that has better pre-list spacing
% when used as the first thing in a \section 
\newenvironment{lonelist}[1][\enskip\textbullet]%
        {\vspace{-\baselineskip}\begin{list}{#1}{%
        \setlength{\partopsep}{0pt}%
        \setlength{\topsep}{0pt}}}
        {\end{list}\vspace{-.6\baselineskip}}

% An itemize-style list with little space between items
\newenvironment{innerlist}[1][\enskip\textbullet]%
        {\begin{compactitem}[#1]}{\end{compactitem}}

% To add some paragraph space between lines.
% This also tells LaTeX to preferably break a page on one of these gaps
% if there is a needed pagebreak nearby.
\newcommand{\blankline}{\quad\pagebreak[2]}

%%%%%%%%%%%%%%%%%%%%%%%% End Helper Commands %%%%%%%%%%%%%%%%%%%%%%%%%%%

%%%%%%%%%%%%%%%%%%%%%%%%% Begin CV Document %%%%%%%%%%%%%%%%%%%%%%%%%%%%

\begin{document}

 
\makeheading{\large{Katherine Ann Scott} }

\section{Contact Information}
%
% NOTE: Mind where the & separators and \\ breaks are in the following
%       table.
%
% ALSO: \rcollength is the width of the right column of the table 
%       (adjust it to your liking; default is 1.85in).
%
\newlength{\rcollength}\setlength{\rcollength}{1.85in}%
%
\begin{tabular}[t]{@{}p{\textwidth-\rcollength}p{\rcollength}}
\href{mailto:katherineAScott@gmail.com}{Katherine Ann Scott}           & \textit{Voice: } (734) 358-0175 \\
\href{http://goo.gl/maps/QI0cn}{1701 Dexter Ave.}                      & \textit {Web: }\href{http://www.kscottz.com}{www.kscottz.com} \\
\href{http://goo.gl/maps/QI0cn}{Ann Arbor, MI, 48103 USA}              & \textit{Twitter: }\href{https://twitter.com/kscottz}{@kscottz} \\
\href{mailto:katherineAScott@gmail.com}{ katherine.a.scott@gmail.com}  & \textit{Github: }\href{https://github.com/kscottz}{kscottz}\\
                                                                       
\end{tabular}

\section{Objectives}
I want to work with a community of researchers to put my computer vision and robotics skills to work building a better world. 

\section{Education}
\href{http://www.engineering.columbia.edu/}{\textbf{Columbia University - School of Engineering and Applied Sciences}} 2011

 Master of Science in 
        \href{http://www.cs.columbia.edu/education/ms/visionAndGraphics}
             {Computer Science} with emphasis on computer vision.\\


\href{http://www.engin.umich.edu/}{\textbf{The University of Michigan College of Engineering}} 2005 

Bachelor of Science and Engineering in both
        \href{http://www.eecs.umich.edu/}
             {Computer and Electrical Engineering}.


\section{Professional \\ Experience}
\href{http://www.sightmachine.com/}{\textbf{SightMachine}} \\
\textit{Vice President of Research and Development}%
        \hfill \textbf{May 2011 to May 2013}  
\begin{outerlist}
\item Developed a significant portion of the \href{http://www.simplecv.org/}{SimpleCV} python library. SimpleCV is an open source python framework for quickly developing computer vision and image processing applications. 
\item Designed custom computer vision applications for industrial and manufacturing customers covering such topics as metrology, image recognition, temporal signal processing, and color space analysis. This process included everything from initial system design and specification all the way through final testing, validation, and deployment.
\item Built and managed the SimpleCV open source community. This included validating user pull requests and bug reports on \href{https://github.com/sightmachine/SimpleCV}{github}, answering question on our \href{http://help.simplecv.org/questions/}{help forum}, managing our \href{http://sourceforge.net/blog/simplecv-gsoc/}{Google Summer of Code students}, co-authoring our \href{http://shop.oreilly.com/product/0636920024057.do}{O'Reilly publication} on SimpleCV, and producing SimpleCV \href{http://www.youtube.com/watch?v=UZSm7Q2bZoc&}{talks} and \href{http://www.youtube.com/watch?v=UZSm7Q2bZoc&}{tutorials}.
\end{outerlist}
\blankline

\href{http://www.cybernet.com/}{\textbf{Cybernet Systems Corporation}}\\
\textit{Research Engineer}%
        \hfill \textbf{February 2005 to December 2010} \\
\begin{outerlist}
\item Wrote the proposal for, won, managed, and contributed significant engineering work to four Phase I Small Business Innovative Research (SBIR) grants and two Phase II SBIR grants totaling nearly two million dollars in research funds.
\item Developed a number of innovative software prototypes in the fields of computer vision, graphics, augmented reality and robotics.  This included a Phase II augmented reality simulation system for PEO-STRI, a tablet augmented reality application for the Marines, and smart CCTV camera systems for the Army. 
\item Worked as a lead engineer, project manager, and primary investigator on various projects and engaged in recruiting, customer relations, business development activities.  

\end{outerlist}

%%%%%%%%%%%%%%%%%%%%%%%%%%%%%%%%% Page break

\section{Technical Skills} 
 I am adept at writing computer vision, graphics, and robotics software in Python, Java, C++, C\#, C and a host of other languages on Linux, Mac, and Windows platforms using a variety of development tools and environments.\\

I have a strong interest in delivering computer vision and graphics applications via the web using WebRTC, JavaScript, and Emscripten/LLVM. I am also highly interested in using Python optimization tools to improve image processing speeds.

\center{\textit{For a complete list of publications, patents, and work history please see my \href{http://www.kscottz.com}{curriculum vitae}.}

%\center{ \textit{ References available upon request. }}
\end{document}


%%%%%%%%%%%%%%%%%%%%%%%%%% End CV Document %%%%%%%%%%%%%%%%%%%%%%%%%%%%%
